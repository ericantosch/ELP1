\documentclass{article}

\usepackage{circuitikz} %Für die Schaltpläne
\usepackage[T1]{fontenc} 
\usepackage[UTF8]{inputenc}
\usepackage{amsmath}
\usepackage{amssymb}
\usepackage{fancyhdr}
\usepackage{graphicx}
\usepackage{hyperref}
\usepackage{tikz}
\usepackage[ngerman]{babel}
    \usetikzlibrary{arrows}
    \usetikzlibrary{arrows.meta,topaths}
    \usetikzlibrary{bending}
    \usetikzlibrary{calc}
\title{Elektrotechnik 1 - Praktikum 1}


\usepackage[
  includehead,
  headheight = 17mm,
  footskip = \dimexpr\headsep+\ht\strutbox\relax,
  tmargin = 0mm,
  bmargin = \dimexpr17mm+2\ht\strutbox\relax,
]{geometry}

\usepackage{anyfontsize}

\usepackage{xcolor}




\pagestyle{fancy}
\fancyhead[L]{\leftmark}
\fancyhead[R]{}
\fancyfoot[L]{}
\fancyfoot[C]{\thepage}
\fancyfoot[R]{\includegraphics[scale=0.2]{assets/images/haw.jpg}}
\renewcommand\headrulewidth{0.5pt}


\begin{document}

\pagestyle{empty}

\begin{tikzpicture}[overlay,remember picture]

  \fill[black!2] (current page.south west) rectangle (current page.north east);

  \begin{scope}[transform canvas ={rotate around ={45:($(current page.north west)+(-.5,-6)$)}}]

    \shade[rounded corners=18pt, left color=green, right color=blue!40] ($(current page.north west)+(-.5,-6)$) rectangle ++(9,1.5);

  \end{scope}

  \begin{scope}[transform canvas ={rotate around ={45:($(current page.north west)+(.5,-10)$)}}]

    \shade[rounded corners=18pt, left color=lightgray,right color=lightgray!60] ($(current page.north west)+(0.5,-10)$) rectangle ++(15,1.5);

  \end{scope}

  \begin{scope}[transform canvas ={rotate around ={45:($(current page.north west)+(0.5,-10)$)}}]

    \shade[rounded corners=8pt, left color=lightgray] ($(current page.north west)+(1.5,-9.55)$) rectangle ++(7,.6);

  \end{scope}

  \begin{scope}[transform canvas ={rotate around ={45:($(current page.north)+(-1.5,-3)$)}}]

    \shade[rounded corners=12pt, left color=orange!80, right color=orange!60] ($(current page.north)+(-1.5,-3)$) rectangle ++(9,0.8);

  \end{scope}

  \begin{scope}[transform canvas ={rotate around ={45:($(current page.north)+(-3,-8)$)}}]

    \shade[rounded corners=28pt, left color=red!80, right color=red!80] ($(current page.north)+(-3,-8)$) rectangle ++(15,1.8);

  \end{scope}

  \begin{scope}[transform canvas ={rotate around ={45:($(current page.north west)+(4,-15.5)$)}}]

    \shade[rounded corners=25pt, left color=orange, right color=blue] ($(current page.north west)+(4,-15.5)$) rectangle ++(30,1.8);

  \end{scope}

  \begin{scope}[transform canvas ={rotate around ={45:($(current page.north west)+(13,-10)$)}},]

    \shade[rounded corners=22pt, left color=blue,right color=green] ($(current page.north west)+(13,-10)$) rectangle ++(15,1.5);

  \end{scope}

  \begin{scope}[transform canvas ={rotate around ={45:($(current page.north west)+(18,-8)$)}},]

    \shade[rounded corners=8pt, left color=lightgray] ($(current page.north west)+(18,-8)$) rectangle ++(15,0.6);

  \end{scope}

  \begin{scope}[transform canvas ={rotate around ={45:($(current page.north west)+(19,-5.65)$)}},]

    \shade[rounded corners=12pt, left color=lightgray] ($(current page.north west)+(19,-5.65)$) rectangle ++(15,0.8);

  \end{scope}

  \begin{scope}[transform canvas ={rotate around ={45:($(current page.north west)+(20,-9)$)}}]

    \shade[rounded corners=20pt, left color=red, right color=red!80] ($(current page.north west)+(20,-9)$) rectangle ++(14,1.2);

  \end{scope}

  \draw[ultra thick,gray] ($(current page.center)+(5,2)$) -- ++(0,-3cm) node[midway,left=0.25cm,text width=5cm,align=right,black!75]{{\fontsize{25}{30} \selectfont \bf Elektronik 1\\[10pt] Praktikum 1}} node[midway,right=0.25cm,text width=6cm,align=left,orange]{{\fontsize{70}{86} \selectfont 2020}};

  \node at ($(current page.center)+(0,-4)$) {{\fontsize{60}{72} \selectfont Passive Bauelemente}};

  \node[text width=8cm,align=center] at ($(current page.center)+(0,-6.5)$) {{\fontsize{16}{20} \selectfont \textcolor{orange}{ \bf \today}} \\[3pt] Florian Tietjen\\[3pt] Eric Antosch};

\end{tikzpicture}
\newpage


\tableofcontents


\newpage

\section{Serieninduktivität eines Drahtwiderstands}

\subsection{Vorbereitung}
\begin{figure}

\end{figure}
Zunächst wollen wir die gegebenen Formeln so umstellen, dass wir mit ihnen die gesuchten Größen errechnen können.

Für unsere erste Schaltung gilt:
\begin{equation}
  f_{1Res} = \frac{1}{2\cdot \pi \cdot \sqrt{L_S \cdot C}}
\end{equation}
Durch ein wenig umstellen erhalten wir dann folgende Form, die wir dann quadrieren:
\begin{align*}
  \sqrt{L_S\cdot C} & = \frac{1}{2\cdot\pi \cdot f_{1Res}}    \\
  L_S \cdot C       & = \frac{1}{4\cdot\pi^2\cdot f_{1Res}^2}
\end{align*}
Wir bringen nun zum Schluss noch $C$ auf die andere Seite und erhalten:
\begin{equation}
  L_S = \frac{1}{4\cdot\pi^2\cdot f_{1Res}^2\cdot C}
\end{equation}
Um nun schlussendlich auch die Parallelresonanzfrequenz zu berechnen, gibt es noch eine entsprechende Formel:
\begin{equation}
  f_{2Res} = \frac{\sqrt{1-\frac{R^2}{L_s/C}}}{2\cdot\pi\cdot\sqrt{L_S\cdot C}}
\end{equation}
\end{document}







